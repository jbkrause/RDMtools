In this document, several useful \textbf{research data management tools are listed and described} for each step of their research throughout the data lifecyle management.

\vspace{0.5cm}

This document is mostly generic from an institutional point of view, however it offers specific information intended for EPFL and Swiss researchers for some targeted points.

\vspace{0.5cm}

\noindent This selection aims to help researchers \textbf{make the most out of their data}, and especially:
\begin{itemize}
\item \textbf{save time} in the long run
\item \textbf{collaborate efficiently } on their data 
\item promote \textbf{reproducible research}
\item enhance the \textbf{visibility of their work} 
\item meet \textbf{funders' data requirements} (Horizon 2020, SNF...) 
\item meet \textbf{publishers' data requirements} (Nature Publishing Group, PLoS...) 
\item \textbf{minimize the risks} around their data (such as data loss or corruption, data leak, etc.)
\item  \textbf{open} or \textbf{secure} their intellectual property privacy
\end{itemize}


